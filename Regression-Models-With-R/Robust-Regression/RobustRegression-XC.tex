\section{Implementation of Robust Regression}
%======================================%
% NOTES


%
%We are going to first use the Huber 
%weights in this example. We will then look at the final weights created by the IRLS process. This can 
%be very useful. 
%Also we will look at an alternative weighting approach to Huber’s weighting – called Bisquare 
%weighting.

In Huber weighting, observations with small residuals get a weight of 1 and the larger the residual, 
the smaller the weight. This is defined by the weight function 

\section{MA4605 Notes}
Robust regression is an alternative to ordinary least squares regression (OLS , the type of regression 
we have studied thus far) when data is contaminated with outliers or influential observations and it 
can also be used for the purpose of detecting influential observations. 



\section{Implementing Robust Regression with R}

When fitting a least squares regression, we might find some outliers or high leverage data points. 
We have decided that these data points are not data entry errors, neither they are from a different 
population than most of our data. So we have no proper reason to exclude them from the analysis. 

Robust regression might be a good strategy since it is a compromise between excluding these points 
entirely from the analysis and including all the data points and treating all them equally in OLS 
regression. The idea of robust regression is to weigh the observations differently based on how well 
behaved these observations are. 



\end{document} 


