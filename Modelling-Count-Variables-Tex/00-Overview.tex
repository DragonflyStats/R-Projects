\documentclass[MASTER.tex]{subfiles}
\begin{document}

\begin{frame}
\Large
\noindent \textbf{Overview}
\begin{enumerate}
\item Introduction to Modelling Count Variables
\item Poisson Regression
\item Negative Binomial Regression
\item Zero-Inflated Models and Vuong Tests
\item Zero Truncation
\end{enumerate}

\end{frame}
\begin{frame}
	%R Data Analysis Examples: Poisson Regression
	
	{\LARGE
		PART 1:	Introduction to Modelling Count Variables
	} \bigskip
	\Large


\end{frame}
%==================================================================%
\begin{frame}
	\Large
\textbf{Introduction}
\begin{itemize}
\item This presentation is about regression methods in which the dependent variable takes
 count (nonnegative integer) values. \smallskip \item The dependent variable is usually the number of times an event occurs in a certain period of time.
\end{itemize}
\end{frame}
%================================================================== %
\begin{frame}[fragile]
\large
	\begin{itemize}
		\item Linear regression is used to model and predict continuous measurement variables.
		\item Poisson regression is used to model and predict discrete count variables.
	\end{itemize}
	\textit{Poisson regression assumes the response variable Y has a Poisson distribution, and assumes the logarithm of its expected value can be modeled by a linear combination of unknown parameters. A Poisson regression model is sometimes known as a log-linear model, especially when used to model contingency tables.}
	
\end{frame}
%================================================================== %
%\begin{frame}[fragile]
%	\textbf{Examples of Poisson regression}
%	\begin{itemize}
%		\item[(1) ]The number of persons killed by mule or horse kicks in the Prussian army per year. Ladislaus Bortkiewicz collected data from 20 volumes of\textit{ Preussischen Statistik}. These data were collected on 10 corps of the Prussian army in the late 1800s over the course of 20 years.
%		
%		\item[(2)] The number of people in line in front of you at the grocery store. Predictors may include the number of items currently offered at a special discounted price and whether a special event (e.g., a holiday, a big sporting event) is three or fewer days away.
%		
%		\item[(3)] The number of awards earned by students at one high school. Predictors of the number of awards earned include the type of program in which the student was enrolled (e.g., vocational, general or academic) and the score on their final exam in math.
%	\end{itemize}
%\end{frame}
%======================================================== %
\begin{frame}
	\frametitle{Overview}
\vspace{-0.5cm}
	\Large Some
examples of \textbf{ event counts } are:
\begin{itemize}
\item number of claims per year on a particular car owner’s insurance policy,
\item number of workdays missed due to sickness of a dependent in a one-year period,
\item number of papers published per year by a researcher.
\end{itemize}
\end{frame}
%================================================================================================%
\begin{frame}[fragile]
\frametitle{Modelling Count Variables } 
\Large	
\textbf{Poisson Distribution} 
	
\begin{itemize}
		\item The number of persons killed by mule or horse kicks in the Prussian army per year. 
		\item Ladislaus Bortkiewicz collected data from 20 volumes of Preussischen Statistik. 
		\item These data were collected on 10 corps of the Prussian army in the late 1800s over the course of 20 years, giving a total of 200 observations of one corps for a one year period. \item The unit period of observation is thus one year. 
\end{itemize}
	
\end{frame}

%======================================================== %
\begin{frame}
\frametitle{Poisson Distribution: Prussian Cavalary}
\Large
	\begin{itemize}
\item The total deaths from horse kicks were 122, and the average number of deaths per year per corps was thus 122/200 = 0.61. 
%\item This is a rate of less than 1. It is also obvious that it is meaningless to ask how many times per year a cavalryman was not killed by the kick of a horse. 
\item In any given year, we expect to observe, well, not exactly 0.61 deaths in one corps 
%(that is not possible; deaths occur in modules of 1), but sometimes none, sometimes one, occasionally two, perhaps once in a while three, and (we might intuitively expect) very rarely any more. 
\item Here, then, is the classic Poisson situation: a rare event, whose average rate is small, with observations made over many small intervals of time.
\end{itemize}
\end{frame}

%======================================================== %
\begin{frame}[fragile]
\frametitle{Generating Random Numbers}
	\begin{framed}
		\begin{verbatim}
		> X <- rpois(200,lambda=0.61)
		> X
		[1]   1 2 0 1 0 3 0 0 1 0 0 4 0 0 0 1 0 1 0 2
		[21]  0 0 0 2 2 0 0 0 1 0 0 0 0 1 0 0 0 1 2 0
		[41]  0 0 1 0 1 0 1 0 0 1 1 0 1 0 0 1 0 0 3 1
		.......
		[141] 0 0 0 0 1 2 0 1 0 1 0 0 0 0 0 0 0 1 0 0
		[161] 1 0 1 0 0 0 0 1 0 0 0 0 0 1 1 1 0 2 0 1
		[181] 0 0 2 0 2 0 0 1 0 0 3 1 0 0 0 1 1 0 0 0
		>
		> mean(X) ;var(X) 
		[1] 0.53
		[1] 0.5317588
		\end{verbatim}
	\end{framed}
\end{frame}
%======================================================== %
\begin{frame}
\frametitle{Poisson Distribution Assumptions}
\Large
\begin{itemize}
\item  Poisson Regression is main technique used to model count variables.

\item Assumption underlying Poisson Distribution : Mean and Variance are equal
\[ \mathrm{E}(X)  = \mathrm{Var}(X) \]

\item Allow for a margin of error of about 5\% . Simulation Studies can be used to determine the validity of this assumption. (see Next Slide)

% \item  Sometimes conventional Poisson Regression is not an appropriate technique, and alternative or variant techniques are used instead. For example, Negative Binomial regression is for modelling ``over-dispersed" count outcome variables.


\end{itemize}
\end{frame}
%=========================================================%
\begin{frame}[fragile]
\frametitle{Simulation Studies}
\large
	\begin{framed}
		\begin{verbatim}
> X=rpois(1000,lambda=1);mean(X);var(X)
[1] 1.001
[1] 1.028027
> X=rpois(5000,lambda=0.5);mean(X);var(X)
[1] 0.5074
[1] 0.5232499
> X=rpois(2500,lambda=0.7);mean(X);var(X)
[1] 0.7248
[1] 0.7317577
> X=rpois(500,lambda=3);mean(X);var(X)
[1] 3.076
[1] 2.851928
		\end{verbatim}
	\end{framed}
\end{frame}
%=========================================================%
\begin{frame}[fragile]
\frametitle{Simulation Studies}
\large
	\begin{framed}
		\begin{verbatim}
> Ratio = numeric()
> M = 10000
> 
> for ( i in 1:M){
+       X=rpois(2500,lambda=5);
+       Ratio[i] = var(X)/mean(X)
+ }
> 
> quantile(Ratio, c(0.025,0.975))
     2.5%     97.5% 
0.9452617 1.0563994 

		\end{verbatim}
	\end{framed}
\end{frame}


%============================================%

\begin{frame}
\frametitle{Problem Areas}
\Large
\begin{description}
\item[Over-Dispersion] : Important Poisson Distributon assumption does not hold 
\[ \mathrm{E}(X)  < \mathrm{Var}(X) \]
\item[Zero-Inflation] : More ``Zeros" would occure than in conventional Poisson Process (This is actually "overdispersion" also, but we will treat them separately). \smallskip
\item[ Zero-Truncation] : Process does not allow for a ``Zero" outcome. 
\end{description}
\end{frame}

%==========================================%
\begin{frame}[fragile]
	
	\frametitle{Poisson Regression with \texttt{R}}
	\Large
	\textbf{Over-Dispersion}\\ 
	\begin{itemize}
		\item Overdispersion is the presence of greater variability
		in a data set than would be expected based on a given simple statistical model.
		\item Violation of Poisson Distribution Assumption:
		\[ \mathrm{Var}(X) >  \mathrm{E}(X) \]
	\end{itemize}
\end{frame}
%==========================================%

%==========================================%
\begin{frame}[fragile]
	\frametitle{Poisson Regression with \texttt{R}}
	\Large
	\textbf{Zero-Inflation}
	\begin{itemize}
		\item One common cause of over-dispersion is excess zeros, which in turn are generated by an additional data generating 
		process. 
		\item In this situation, zero-inflated model should be considered.
		\item If the data generating process does not allow for any 0s (such as the number of days spent in the hospital), then a zero-truncated model may be more appropriate.
	\end{itemize}
\end{frame}
%==========================================%
\begin{frame}[fragile]
\frametitle{Poisson Regression with \texttt{R}}
	\Large
	\textbf{Over-Dispersion}
	\begin{itemize}
		
		\item When there seems to be an issue of dispersion, we should first check if our model is appropriately specified, 
		such as omitted variables and functional forms. 
		\item For example, if we omitted the predictor variable prog in the example above, our model would seem to have a 
		problem with over-dispersion. 
		\item In other words, a misspecified model could present a symptom like an over-dispersion problem. 
	\end{itemize}
\end{frame}


%==========================================%
\begin{frame}[fragile]
	\frametitle{Poisson Regression with \texttt{R}}
	\Large
	\begin{itemize}
		\item Assuming that the model is correctly specified, the assumption that the conditional variance is equal 
		to the conditional mean should be checked. 
		\item There are several tests including the likelihood ratio test of over-dispersion parameter alpha by running the 
		same model using negative binomial distribution. 
%		\item The \texttt{R} package \textbf{pscl} (Political Science Computational Laboratory, Stanford University) provides many functions for binomial 
%		and count data including \texttt{odTest} for testing over-dispersion. 
	\end{itemize}
\end{frame}
%=========================================================== %
\begin{frame}[fragile]
\frametitle{Generalized Linear Models}\large
\textbf{The \texttt{glm()} function}
\begin{itemize}
\item In statistics, the problem of modelling count variables is an example of generalized linear modelling.
\item Generalized linear models are fit using the \texttt{glm()} function. 
\item The form of the \texttt{glm} function is
\end{itemize}
{
\normalsize
\begin{framed}
 \begin{verbatim}
glm( modelformula, 
       family=familytype(link=linkfunction),
       data=dataname)
\end{verbatim}
\end{framed}
}
\end{frame}

%========================================================== %
\begin{frame}[fragile]
	\frametitle{Generalized Linear Models}
	\Large
\begin{center}
\begin{tabular}{|c|l|} \hline
	Family &	Default Link Function \\ \hline \hline 
	binomial&	\texttt{(link = "logit")} \\ \hline
	gaussian&	\texttt{(link = "identity")} \\ \hline
	Gamma&	\texttt{(link = "inverse")} \\ \hline
	inverse.gaussian&	\texttt{(link = "$1/mu^2$")} \\ \hline
	\textbf{poisson}	&\texttt{(link = "log")} \\ \hline
	%quasi&	\texttt{(link = "identity", variance = "constant")} \\ \hline
	quasibinomial&	\texttt{(link = "logit")} \\ \hline
	quasipoisson&	\texttt{(link = "log")} \\ \hline
	
\end{tabular}
\end{center}
\end{frame}
%========================================================== %
\begin{frame}
\frametitle{Generalized Linear Models}
	\large
	\textbf{Texts on GLMs}\\ \bigskip
\begin{itemize}
\item Dobson, A. J. (1990) An Introduction to Generalized Linear Models. (\textit{London: Chapman and Hall}.)
\bigskip
\item Hastie, T. J. and Pregibon, D. (1992) Generalized linear models. Chapter 6 of Statistical Models in S eds J. M. Chambers and T. J. Hastie, Wadsworth \& Brooks/Cole.
\bigskip
\item McCullagh P. and Nelder, J. A. (1989) Generalized Linear Models. (\textit{London: Chapman and Hall.})
\bigskip
\item Venables, W. N. and Ripley, B. D. (2002) Modern Applied Statistics with S. \textit{New York: Springer.}
\end{itemize}

\end{frame}


%========================================================== %
\begin{frame}
	\textbf{pscl: Political Science Computational Laboratory} %, Stanford University

\begin{description}
	\item[Author(s):] Simon Jackman et al (Stanford University)
	%\item[License:] GPL-2
	\item[URL:] http://pscl.stanford.edu/
\end{description}\bigskip

\textbf{Description}\\
Bayesian analysis of item-response theory (IRT) models, roll call analysis; computing highest density regions; maximum likelihood estimation of zero-inflated and hurdle models for count data; goodness-of-fit measures for GLMs; data sets used in writing and teaching at the Political Science Computational Laboratory; seats-votes curves.
\end{frame}

%========================================================== %
\begin{frame}
	\textbf{glm2: Fitting Generalized Linear Models}
\begin{description}
	\item[Author(s):] Ian Marschner 

\end{description}
Fits generalized linear models using the same model specification as glm in the stats package, but with a modified default fitting method that provides greater stability for models that may fail to converge using glm
\end{frame}


\begin{frame}
	\textbf{VGAM: Vector Generalized Linear and Additive Models}
\begin{description}
\item[Author(s):] Thomas W. Yee (t.yee@auckland.ac.nz)
%\item[License:] GPL-2
\item[URL:] http://www.stat.auckland.ac.nz/$\sim$ yee/VGAM
\end{description}\bigskip
Vector generalized linear and additive models, and associated models (Reduced-Rank VGLMs, Quadratic RR-VGLMs, Reduced-Rank VGAMs). \\ \bigskip This package fits many models and distribution by maximum likelihood estimation (MLE) or penalized MLE. Also fits constrained ordination models in ecology.
\end{frame}


%================================================================== %
\begin{frame}
	\large
\noindent \textbf{MA4128  - Review Questions} \smallskip
\begin{itemize}
\item[(i)] Describe Event Counts / Count Variables.
\item[(ii)] Poisson Distribution : Asummption of Parameter Equality what is it? how to check?
\item[(iii)] State the three cases where assumption does not hold
\end{itemize}
\smallskip
You are not required to know anything about \texttt{R} implementation.
\end{frame}

\end{document}






