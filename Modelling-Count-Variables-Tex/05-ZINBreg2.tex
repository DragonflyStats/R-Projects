\documentclass[MASTER.tex]{subfiles}
\begin{document} 
\begin{frame}
\frametitle{Vuong Testing}
	\begin{itemize}
\item Note that the model output above does not indicate in any way if our zero-inflated model is an improvement over a standard negative binomial regression. 
\item We can determine this by running the corresponding standard negative binomial model and then performing a Vuong test of the two models.
\item We use the MASS package to run the standard negative binomial regression.
	\end{itemize}


\end{frame}
%============================================================================== %
\begin{frame}[fragile]
	\frametitle{Vuong Testing}
\begin{verbatim}
library(MASS)
summary(m2 <- glm.nb(count ~ child + camper, data = zinb))
.....
\end{verbatim}
\end{frame}
%============================================================================== %
\begin{frame}[fragile]
	\frametitle{Vuong Testing}
\begin{verbatim}
## Coefficients:
##             Estimate Std. Error z value Pr(>|z|)    
## (Intercept)    1.073      0.242    4.42  9.7e-06 ***
## child         -1.375      0.196   -7.03  2.1e-12 ***
## camper1        0.909      0.284    3.21   0.0013 ** 
## ---
## Signif. codes:  0 '***' 0.001 '**' 0.01 '*' 0.05 '.' 0.1 ' ' 1
\end{verbatim}
\end{frame}
%%============================================================================== %
%\begin{frame}[fragile]
%	\begin{verbatim}
%## (Dispersion parameter for Negative Binomial(0.2553) family taken to be 1)
%## 
%##     Null deviance: 258.93  on 249  degrees of freedom
%## Residual deviance: 201.89  on 247  degrees of freedom
%## AIC: 887.4
%\end{verbatim}
%\end{frame}
%%============================================================================== %
%\begin{frame}[fragile]
%	\begin{verbatim}
%## Number of Fisher Scoring iterations: 1
%## 
%## 
%##               Theta:  0.2553 
%##           Std. Err.:  0.0329 
%## 
%##  2 x log-likelihood:  -879.4210
%\end{verbatim}
%\end{frame}
%============================================================================== %
\begin{frame}[fragile]
\frametitle{Vuong Testing}
	\begin{verbatim}
vuong(m1, m2)
 
## Vuong Non-Nested Hypothesis Test-Statistic: 1.702 
## (test-statistic is asymptotically distributed N(0,1) under the
##  null that the models are indistinguishible)
## in this case:
## model1 > model2, with p-value 0.0444
\end{verbatim}
\end{frame}
%========================================================================== %

%========================================================================== %
\begin{frame}
\begin{itemize}
\item The log odds of being an excessive zero would decrease by 1.67 for every additional person in the group. 
\item In other words, the more people in the group the less likely that the zero would be due to not gone fishing. 
\item Put plainly, the larger the group the person was in, the more likely that the person went fishing.
\item The Vuong test suggests that the zero-inflated negative binomial model is a significant improvement over a standard negative binomial model. 
\end{itemize}
\end{frame}
\end{document}