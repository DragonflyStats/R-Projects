%======================================================================================%

% Ordinal Logistic

Things to consider
•Perfect prediction: Perfect prediction means that one value of a predictor variable is associated with only one value of the response variable. If this happens, Stata will usually issue a note at the top of the output and will drop the cases so that the model can run.
•Sample size: Both ordered logistic and ordered probit, using maximum likelihood estimates, require sufficient sample size. How big is big is a topic of some debate, but they almost always require more cases than OLS regression.
•Empty cells or small cells: You should check for empty or small cells by doing a crosstab between categorical predictors and the outcome variable. If a cell has very few cases, the model may become unstable or it might not run at all.
•Pseudo-R-squared: There is no exact analog of the R-squared found in OLS. There are many versions of pseudo-R-squares. Please see Long and Freese 2005 for more details and explanations of various pseudo-R-squares.
•Diagnostics: Doing diagnostics for non-linear models is difficult, and ordered logit/probit models are even more difficult than binary models.


%======================================================================================%

% Multinomial Logistic

Things to consider
•The Independence of Irrelevant Alternatives (IIA) assumption: Roughly, the IIA assumption means that adding or deleting alternative outcome categories does not affect the odds among the remaining outcomes. There are alternative modeling methods, such as alternative-specific multinomial probit model, or nested logit model to relax the IIA assumption.
• Diagnostics and model fit: Unlike logistic regression where there are many statistics for performing model diagnostics, it is not as straightforward to do diagnostics with multinomial logistic regression models. For the purpose of detecting outliers or influential data points, one can run separate logit models and use the diagnostics tools on each model. 
•Sample size: Multinomial regression uses a maximum likelihood estimation method, it requires a large sample size. It also uses multiple equations. This implies that it requires an even larger sample size than ordinal or binary logistic regression.
•Complete or quasi-complete separation: Complete separation means that the outcome variable separate a predictor variable completely, leading perfect prediction by the predictor variable.
•Perfect prediction means that only one value of a predictor variable is associated with only one value of the response variable. But you can tell from the output of the regression coefficients that something is wrong. You can then do a two-way tabulation of the outcome variable with the problematic variable to confirm this and then rerun the model without the problematic variable.
•Empty cells or small cells: You should check for empty or small cells by doing a cross-tabulation between categorical predictors and the outcome variable. If a cell has very few cases (a small cell), the model may become unstable or it might not even run at all.


%======================================================================================%

% Poisson

Things to consider
 •When there seems to be an issue of dispersion, we should first check if our model is appropriately specified, such as omitted variables and functional forms. For example, if we omitted the predictor variable prog in the example above, our model would seem to have a problem with over-dispersion. In other words, a misspecified model could present a symptom like an over-dispersion problem. 
•Assuming that the model is correctly specified, the assumption that the conditional variance is equal to the conditional mean should be checked. There are several tests including the likelihood ratio test of over-dispersion parameter alpha by running the same model using negative binomial distribution. R package pscl (Political Science Computational Laboratory, Stanford University) provides many functions for binomial and count data including odTest for testing over-dispersion. 
•One common cause of over-dispersion is excess zeros, which in turn are generated by an additional data generating process. In this situation, zero-inflated model should be considered.
•If the data generating process does not allow for any 0s (such as the number of days spent in the hospital), then a zero-truncated model may be more appropriate.
•Count data often have an exposure variable, which indicates the number of times the event could have happened. This variable should be incorporated into a Poisson model with the use of the offset option.
•The outcome variable in a Poisson regression cannot have negative numbers, and the exposure cannot have 0s.
•Many different measures of pseudo-R-squared exist. They all attempt to provide information similar to that provided by R-squared in OLS regression, even though none of them can be interpreted exactly as R-squared in OLS regression is interpreted. For a discussion of various pseudo-R-squares, see Long and Freese (2006) or our FAQ page What are pseudo R-squareds?.
•Poisson regression is estimated via maximum likelihood estimation. It usually requires a large sample size. 


%======================================================================================%

% Negative binomial regression

Things to consider
 •It is not recommended that negative binomial models be applied to small samples. 
•One common cause of over-dispersion is excess zeros by an additional data generating process. In this situation, zero-inflated model should be considered.
•If the data generating process does not allow for any 0s (such as the number of days spent in the hospital), then a zero-truncated model may be more appropriate.
•Count data often have an exposure variable, which indicates the number of times the event could have happened. This variable should be incorporated into your negative binomial regression model with the use of the offset option. See the glm documentation for details.
•The outcome variable in a negative binomial regression cannot have negative numbers.
•You will need to use the m1$resid command to obtain the residuals from our model to check other assumptions of the negative binomial model (see Cameron and Trivedi (1998) and Dupont (2002) for more information).

Things to consider
%======================================================================================%

% Zero Inflated Negative binomial regression 
Here are some issues that you may want to consider in the course of your research analysis.
 •Question about the over-dispersion parameter is in general a tricky one. A large over-dispersion parameter could be due to a miss-specified model or could be due to a real process with over-dispersion. Adding an over-dispersion problem does not necessarily improve a miss-specified model. 
•The zero inflated negative binomial model has two parts, a negative binomial count model and the logit model for predicting excess zeros, so you might want to review these Data Analysis Example pages, Negative Binomial Regression and Logit Regression. 
•Since zero inflated negative binomial has both a count model and a logit model, each of the two models should have good predictors. The two models do not necessarily need to use the same predictors. 
•Problems of perfect prediction, separation or partial separation can occur in the logistic part of the zero-inflated model. 
•Count data often use exposure variable to indicate the number of times the event could have happened. You can incorporate exposure (also called an offset) into your model by using the offset() function. 
•It is not recommended that zero-inflated negative binomial models be applied to small samples. What constitutes a small sample does not seem to be clearly defined in the literature. 
•Pseudo-R-squared values differ from OLS R-squareds, please see FAQ: What are pseudo R-squareds? for a discussion on this issue.
