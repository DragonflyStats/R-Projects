\documentclass[MASTER.tex]{subfiles}
\begin{document}
	
\begin{frame}
%===================================================================================%
\frametitle{Poisson Regression "Exposure" and offset}
\begin{itemize}
\item Poisson regression may also be appropriate for rate data, where the rate is a count of events occurring to a particular unit of observation, divided by some measure of that unit's exposure. 
\item For example, biologists may count the number of tree species in a forest, and the rate would be the number of species per square kilometre. 
\item Demographers may model death rates in geographic areas as the count of deaths divided by person−years. 
\item More generally, event rates can be calculated as events per unit time, which allows the observation window to vary for each unit. 
\end{itemize}
\end{frame}
%============================================================================%
\begin{frame}
	\frametitle{Poisson Regression : Exposure and Offset}
	\Large
In these examples, exposure is respectively unit area, person−years and unit time. In Poisson regression this is handled as an offset, where the exposure variable enters on the right-hand side of the equation, but with a parameter estimate (for log(exposure)) constrained to 1.
\[\log{(\operatorname{E}(Y\mid x))} = \log{(\text{exposure})} + \theta' x\]
which implies
\[\log{(\operatorname{E}(Y\mid x))} - \log{(\text{exposure})} = 
       \log{\left(\frac{\operatorname{E}(Y\mid x)}{\text{exposure}}\right)} = \theta' x\]
\end{frame}
%============================================================================%
\begin{frame}[fragile]
\frametitle{Poisson Regression : Exposure and Offset}
\Large
Offset in the case of a GLM in R can be achieved using the offset() function:

\begin{framed}
\begin{verbatim}
glm(y ~ offset(log(exposure)) + x, family=poisson(link=log) )
\end{verbatim}
\end{framed}

\end{frame}
%============================================================================%
\end{document}