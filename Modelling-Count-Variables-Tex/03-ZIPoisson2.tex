\documentclass[MASTER.tex]{subfiles}
\begin{document}
%\subsection*{Analysis of Deviance Procedure}
%====================================================================================================== %

\begin{frame}[fragile]
\frametitle{Zero-Inflation}
	\large
	\begin{verbatim}
	confint(m1)
	##                     2.5 %  97.5 %
	## count_(Intercept)  1.4302  1.7655
	## count_child       -1.2388 -0.8469
	## count_camper1      0.6505  1.0175
	## zero_(Intercept)   0.5647  2.0302
	## zero_persons      -0.8838 -0.2449
	\end{verbatim}
\end{frame}
%========================================================================= %
\begin{frame}[fragile]
\frametitle{Zero-Inflation}
\Large
	\begin{itemize}
		\item	All of the predictors in both the count and inflation portions of the model are statistically significant. 
		\item We can use other methods  (for example non parametric methods) to compute CIs ( probably more accurate )
%		\item This model fits the data significantly better than the null model, i.e., the intercept-only model. 
%		\item To show that this is the case, we can compare with the current model to a null model without predictors using chi-squared test on the difference of log likelihoods.
	\end{itemize}
\end{frame}

%====================================================================================================== %
%\begin{frame}[fragile]
%	\begin{verbatim}
%	mnull <- update(m1, . ~ 1)
%	
%	pchisq(2 * (logLik(m1) - logLik(mnull)), df = 3, lower.tail = FALSE)
%	## 'log Lik.' 4.041e-41 (df=5)
%	\end{verbatim}
%	
%	\begin{itemize}
%\item Since we have three predictor variables in the full model, the degrees of freedom for the chi-squared test is 3. 
%\item This yields a high significant p-value; thus, our overall model is statistically significant.
%\end{itemize}
%\end{frame}
%=================================================================== %
\begin{frame}
	%R Data Analysis Examples: Poisson Regression
	
	{\LARGE
		PART 4A:	Vuong Test for Zero-Inflation
	} \bigskip
	\Large
	

\end{frame}
%====================================================================================================== %
\begin{frame}[fragile]
\frametitle{Vuong Testing}
	\begin{itemize}
		\item Note that the model output above does not indicate in any way if our zero-inflated model is an improvement over a standard Poisson regression. 
		\item We can determine this by running the corresponding standard Poisson model and then performing a \textbf{Vuong Test} of the two models.
	\end{itemize}
	\begin{verbatim}
	summary(p1 <- glm(count ~ child + camper, 
	family = poisson, data = fishing))
	\end{verbatim}
\end{frame}
%====================================================================================================== %
\begin{frame}[fragile]
	\frametitle{Vuong Testing}
	\large
	\begin{itemize}
		\item The Vuong test compares the zero-inflated model with an ordinary Poisson regression model. 
		\item In this example, we can see that our test statistic is significant, indicating that the zero-inflated model is superior to the standard Poisson model.
	\end{itemize}
\end{frame}
%============================================================================================ %
\begin{frame}[fragile]

	\begin{verbatim}
	vuong(p1, m1)
	
	## Vuong Non-Nested Hypothesis Test-Statistic: -3.574 
	##
	## (test-statistic is asymptotically distributed N(0,1) 
	## under the null hypothesis that the models are 
	## indistinguishible)
	## in this case:
	##
	## model2 > model1, with p-value 0.0001756
	\end{verbatim}
\end{frame}
\end{document}