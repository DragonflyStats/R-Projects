
ABSTRACT
Real-life count data are frequently characterized by overdispersion and excess zeros. Zero-inflated count models
provide a parsimonious yet powerful way to model this type of situation. Such models assume that the data are a
mixture of two separate data generation processes: one generates only zeros, and the other is either a Poisson or
a negative binomial data-generating process. The result of a Bernoulli trial is used to determine which of the two
processes generates an observation.
OVERVIEW
The COUNTREG (count regression) procedure analyzes regression models in which the dependent variable takes
nonnegative integer or count values. The dependent variable is usually the number of times an event occurs. Some
examples of event counts are:
 number of claims per year on a particular car owner’s auto insurance policy
 number of workdays missed due to sickness of a dependent in a 4-week period
 number of papers published per year by a researcher
In count regression, the conditional mean E.yi jxi / of the dependent variable, yi , is assumed to be a function of a vector
of covariates, xi . Possible covariates for the auto insurance example are:
 age of the driver
 type of car
 daily commuting distance

MARGINAL EFFECTS IN COUNT REGRESSION
Marginal effects provide a way to measure the effect of each covariate on the dependent variable. The marginal effect
of one covariate is the expected instantaneous rate of change in the dependent variable as a function of the change
in that covariate, while keeping all other covariates constant. Unlike in linear models, the derivative of the conditional
expectation with respect to xi;j is no longer equal to ?j —that is, @E.yi jxi /=@xi;j ¤ ?j . For example, for the Poisson
regression with E.yi jxi / D ex0

Therefore the marginal effect of the change in covariate xi;j depends not only on ?j , but also on all other estimated
coefficients, and on all other covariate values. Another interpretation is that a one-unit change in the j th covariate leads
to a proportional change in the conditional mean E.yi jxi / of ?j .
BASIC MODELS: POISSON AND NEGATIVE BINOMIAL REGRESSION MODELS
The Poisson (log-linear) regression model is the most basic model that explicitly takes into account the nonnegative
integer-valued aspect of the dependent count variable. In this model, the probability of an event count yi , given the
vector of covariates xi , is given by the Poisson distribution:
