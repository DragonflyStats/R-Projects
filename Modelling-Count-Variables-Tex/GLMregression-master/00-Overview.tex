\documentclass[00-GLMregslides.tex]{subfiles}
\begin{document}
%==================================================================%
	\begin{frame}
		\LARGE
		\textbf{Overview}
	\end{frame}
\begin{frame}
\Large
\begin{itemize}
\item Binary logistic regression, also called a logit model, is used to model \textbf{\textit{dichotomous outcome}} variables. In the logit model the log odds of the outcome is modeled as a linear combination of the predictor variables.
\item Multinomial logistic regression is used to model nominal outcome variables (categorical variables). Again the log odds of the outcomes are modeled as a linear combination of the predictor variables.

\end{itemize}
\end{frame}
%==================================================================%
\begin{frame}
\Large
\begin{itemize}
	\item Ordinal logistic regression is used to model nominal outcome variables, where a hierarchy within categories exists.
	
\item  Poisson regression is used to model count variables.
\item  Negative binomial regression is for modeling count variables, usually for over-dispersed count outcome variables.
\end{itemize}
\end{frame}
\end{document}






