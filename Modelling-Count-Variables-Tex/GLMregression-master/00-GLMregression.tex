\documentclass[]{article}
%======================================================================================%

% Ordinal Logistic

Analysis methods you might consider
 
Below is a list of some analysis methods you may have encountered. Some of the methods listed are quite reasonable while others have either fallen out of favor or have limitations.
•Ordered logistic regression: the focus of this page.
•OLS regression: This analysis is problematic because the assumptions of OLS are violated when it is used with a non-interval outcome variable.
•ANOVA: If you use only one continuous predictor, you could "flip" the model around so that, say, gpa was the outcome variable and apply was the predictor variable. Then you could run a one-way ANOVA. This isn't a bad thing to do if you only have one predictor variable (from the logistic model), and it is continuous.
•Multinomial logistic regression: This is similar to doing ordered logistic regression, except that it is assumed that there is no order to the categories of the outcome variable (i.e., the categories are nominal). The downside of this approach is that the information contained in the ordering is lost.
•Ordered probit regression: This is very, very similar to running an ordered logistic regression. The main difference is in the interpretation of the coefficients.

%======================================================================================%

% Multinomial Logistic

Analysis methods you might consider

•Multinomial logistic regression, the focus of this page. 
•Multinomial probit regression, similar to multinomial logistic regression with independent normal error terms.
•Multiple-group discriminant function analysis. A multivariate method for multinomial outcome variables
•Multiple logistic regression analyses, one for each pair of outcomes: One problem with this approach is that each analysis is potentially run on a different sample. The other problem is that without constraining the logistic models, we can end up with the probability of choosing all possible outcome categories greater than 1.
•Collapsing number of categories to two and then doing a logistic regression: This approach suffers from loss of information and changes the original research questions to very different ones. 
•Ordinal logistic regression: If the outcome variable is truly ordered and if it also satisfies the assumption of proportional odds, then switching to ordinal logistic regression will make the model more parsimonious.
•Alternative-specific multinomial probit regression, which allows different error structures therefore allows to relax the IIA assumption. This requires that the data structure be choice-specific.
•Nested logit model, another way to relax the IIA assumption, also requires the data structure be choice-specific. 

%======================================================================================%

% Poisson

Analysis methods you might consider

Below is a list of some analysis methods you may have encountered. Some of the methods listed are quite reasonable, while others have either fallen out of favor or have limitations.
 •Poisson regression - Poisson regression is often used for modeling count data. Poisson regression has a number of extensions useful for count models.
•Negative binomial regression - Negative binomial regression can be used for over-dispersed count data, that is when the conditional variance exceeds the conditional mean. It can be considered as a generalization of Poisson regression since it has the same mean structure as Poisson regression and it has an extra parameter to model the over-dispersion. If the conditional distribution of the outcome variable is over-dispersed, the confidence intervals for Negative binomial regression are likely to be narrower as compared to those from a Poisson regression.
•Zero-inflated regression model - Zero-inflated models attempt to account for excess zeros. In other words, two kinds of zeros are thought to exist in the data, "true zeros" and "excess zeros". Zero-inflated models estimate two equations simultaneously, one for the count model and one for the excess zeros.
•OLS regression - Count outcome variables are sometimes log-transformed and analyzed using OLS regression. Many issues arise with this approach, including loss of data due to undefined values generated by taking the log of zero (which is undefined) and biased estimates.


%======================================================================================%

% Negative binomial regression

Analysis methods you might consider

Before we show how you can analyze this with a zero-inflated negative binomial analysis, let's consider some other methods that you might use.
 •OLS Regression - You could try to analyze these data using OLS regression. However, count data are highly non-normal and are not well estimated by OLS regression.
•Zero-inflated Poisson Regression - Zero-inflated Poisson regression does better when the data is not overdispersed, i.e. when variance is not much larger than the mean. 
•Ordinary Count Models - Poisson or negative binomial models might be more appropriate if there are not excess zeros.

\end{document}
