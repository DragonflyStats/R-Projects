Influence Diagnostics
 
% - http://support.sas.com/documentation/cdl/en/statug/63033/HTML/default/viewer.htm#statug_mixed_sect024.htm

%---------------------------------------------------------------------------------------------%

Basic Idea and Statistics
 
The general idea of quantifying the influence of one or more observations relies on computing parameter estimates based on all data points, removing the cases in question from the data, refitting the model, and computing statistics based on the change between full-data and reduced-data estimation. Influence statistics can be coarsely grouped by the aspect of estimation that is their primary target: 

\begin{itemize}	
\item	influence on fitted and predicted values: PRESS residual, PRESS statistic (Allen 1974), DFFITS (Belsley, Kuh, and Welsch 1980, p. 15) 
\item	influence on parameter estimates: Cook’s (Cook 1977, 1979), MDFFITS (Belsley, Kuh, and Welsch 1980, p. 32) 
\item	influence on precision of estimates: CovRatio and CovTrace 
\item	outlier properties: internally and externally studentized residuals, leverage 
\item	overall measures compare changes in objective functions: (restricted) likelihood distance (Cook and Weisberg 1982, Ch. 5.2) 
\end{itemize}	

%---------------------------------------------------------------------------------------------%

For linear models for uncorrelated data, it is not necessary to refit the model after removing a data point in order to measure the impact of an observation on the model. The change in fixed effect estimates, residuals, residual sums of squares, and the variance-covariance matrix of the fixed effects can be computed based on the fit to the full data alone. By contrast, in mixed models several important complications arise. Data points can affect not only the fixed effects but also the covariance parameter estimates on which the fixed-effects estimates depend. Furthermore, closed-form expressions for computing the change in important model quantities might not be available. 

This section provides background material for the various influence diagnostics available with the MIXED procedure. See the section Mixed Models Theory for relevant expressions and definitions. The parameter vector denotes all unknown parameters in the and matrix. 

The observations whose influence is being ascertained are represented by the set and referred to simply as "the observations in ." The estimate of a parameter vector, such as , obtained from all observations except those in the set is denoted . In case of a matrix , the notation represents the matrix with the rows in removed; these rows are collected in . If is symmetric, then notation implies removal of rows and columns. The vector comprises the responses of the data points being removed, and is the variance-covariance matrix of the remaining observations. When , lowercase notation emphasizes that single points are removed, such as . 
