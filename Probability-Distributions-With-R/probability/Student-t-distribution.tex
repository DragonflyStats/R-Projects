\section{Student's $t-$distribution (1)}
\begin{itemize}
\item Student's $t-$distribution is a variation of the normal distribution, designed to factor in the increased uncertainty resulting from smaller samples.
\item The distribution is really a family of distributions, with
a somewhat different distribution associated with the degrees of freedom ($df$). For a confidence interval for the
population mean based on a sample of size n, $df = n - 1$.

\end{itemize}
\begin{verbatim}

> qt(0.975, df=4:28)
[1] 2.776445 2.570582 2.446912 2.364624 2.306004
[6] 2.262157 2.228139 2.200985 2.178813 2.160369
[11] 2.144787 2.131450 2.119905 2.109816 2.100922
[16] 2.093024 2.085963 2.079614 2.073873 2.068658
[21] 2.063899 2.059539 2.055529 2.051831 2.048407
\end{verbatim}




%------------------------------------------------------------------------------%

\textbf{Student's $t-$distribution (2)}
\textbf{IMPORTANT}
\begin{itemize}
\item  With increasing
sample size, the $t-$distribution approaches the form of the standard normal (`Z') distribution.
\item  In fact the standard normal distribution can be thought of as the $t-$distribution with $\infty$ degrees of freedom.

\item  \textbf{REALLY IMPORTANT} For values of $n$ greater then 30, the difference between using $df = n-1$ and $df = \infty$ is negligible.
\item  For computing quantiles, we will consider the `Z' distribution in this way, i.e. using the $t$ distribution instead.
\item  As this will be relevant later, remember that a confidence interval is a \textbf{two-tailed} procedure, i.e. $k=2$.
\end{itemize}


%-----------------------------------------------------------%

%
%\textbf{Student's $t-$distribution (3)}
%
%\begin{itemize}
%\item  Student's t- values are determined using the \texttt{t} family of commands (e.g. \texttt{qt, pt, dt}).
%\item  To compute quantiles, use the code below.
%\item  The degrees of freedom must be additionally be specified. Degrees of freedom are computed as sample size minus one ($n-1$)
%\item  As the degrees of freedom gets larger and larger, the student t distribution converges to the Z distribution.
%
%\end{itemize}
%
%


%-----------------------------------------------------------%


\subsection{Student's $t-$distribution (3)}

\begin{itemize}
\item Student's t- values are determined using the \texttt{t} family of commands (e.g. \texttt{qt, pt, dt}).
\item To compute quantiles, use the code below.
\item The degrees of freedom must be additionally be specified. Degrees of freedom are computed as sample size minus one ($n-1$)
\item As the degrees of freedom gets larger and larger, the student t distribution converges to the Z distribution.

\end{itemize}
\end{document}
