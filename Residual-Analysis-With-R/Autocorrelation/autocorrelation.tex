\documentclass[residuals.tex]{subfiles}

% Load any packages needed for this document
\begin{document}
\subsection*{Autocorrelation} 
\begin{itemize}
 =  Adjacent residuals should not be correlated with each other (\textbf{autocorrelation}). 
 =  If you can use one residual to predict the next residual, there is some predictive information present that is not captured by the predictors. 
 =  Typically, this situation involves time-ordered observations. 
\end{itemize}

\begin{figure}[h!]
\centering
\includegraphics[width=0.7\linewidth]{autocorrelation1}
\caption{(disregard the titles)}

\end{figure}
\begin{itemize}
 =  For example, if a residual is more likely to be followed by another residual that has the same sign, adjacent residuals are positively correlated. 
 =  You can include a variable that captures the relevant time-related information, or use a time series analysis.
\end{itemize}
 
\newpage
\subsection*{Durbin-Watson Test for Autocorrelated Errors}
The \textbf{\textit{Durbin-Watson} }procedure is commonly used to to test for autocorrelation of residuals. To perform this test, we use the \texttt{durbinWatsonTest()} from the car R package. All you have to do is to specify the name of the fitted mode.

\begin{framed}
\begin{verbatim}

 FitMod <- lm(mpg~wt+cyl,data=mtcars)

# library(car)
durbinWatsonTest(FitMod)

\end{verbatim}
\end{framed}
\begin{itemize}
 =  The null hypothesis can simply be stated as "There is no autocorrelation present in the residuals".. 
 =  The \texttt{R} code output provides a $p-$value to base a determination on.
\end{itemize}

\begin{framed}
\begin{verbatim}
> durbinWatsonTest(FitMod)
 lag Autocorrelation D-W Statistic p-value
   1       0.1302185      1.671096   0.252
 Alternative hypothesis: rho != 0
\end{verbatim}
\end{framed}

% http://polisci.msu.edu/jacoby/icpsr/regress3/lectures/week3/11.Outliers.pdf

\end{document}