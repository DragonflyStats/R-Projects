
\begin{document}
\section{Introduction to Residuals}

The difference between the observed value of the dependent variable (y) and the predicted value ($\hat{y}$) is called the \textbf{residual} (e). Each data point has one residual.

\[\mbox{Residual} = \mbox{Observed value} - \mbox{Predicted value}\] 
\[e = y - \hat{y}\]

Both the sum and the mean of the residuals are equal to zero. 
%That is, Σ e = 0 and e = 0.


\newpage
		
%Random pattern	Non-random: U-shaped	Non-random: Inverted U
In the next lesson, we will work on a problem, where the residual plot shows a non-random pattern. And we will show how to "transform" the data to use a linear model with nonlinear data.

%----------------------------------------------------------------------------------------------%



Identifying and fixing the problem so that the predictors now explain the information that they missed before should produce a good-looking set of residuals!

In addition to the above, here are two more specific ways that predictive information can sneak into the residuals:

The residuals should not be correlated with another variable. If you can predict the residuals with another variable, that variable should be included in the model. In Minitab’s regression, you can plot the residuals by other variables to look for this problem.

\newpage
\section{Leverage and Influence}
% http://polisci.msu.edu/jacoby/icpsr/regress3/lectures/week3/11.Outliers.pdf

%%%%%%%%%%%%%%%%%%%%%%%%%%%%%%%%%%%%%%%%%%%%%%%%%%%%%%%%%%%%%%%%%%%%%%%%%%%%%%%%%
\newpage
\section{Diagnostic Plots for Linear Models with \texttt{R}}
Plot Diagnostics for an \texttt{lm} Object


\begin{itemize}
\item
The \textbf{Scale-Location} plot, also called ‘Spread-Location’ or ‘S-L’ plot, takes the square root of the absolute residuals in order to diminish skewness (sqrt(|E|)) is much less skewed than | E | for Gaussian zero-mean E).

\item
The \textbf{Residual-Leverage} plot shows contours of equal Cook's distance, for values of cook.levels (by default 0.5 and 1) and omits cases with leverage one with a warning. If the leverages are constant (as is typically the case in a balanced aov situation) the plot uses factor level combinations instead of the leverages for the x-axis. (The factor levels are ordered by mean fitted value.)
\end{itemize}
\begin{framed}
\begin{verbatim}
par(mfrow=c(4,1))
plot(fittedmodel)
par(opar)
\end{verbatim}
\end{framed}
%%%%%%%%%%%%%%%%%%%%%%%%%%%%%%%%%%%%%%%%%%%%%%%%%%%%%%%%%%%%%%%%%%%%%%%%%%%%%%%%%
%-------------------------------------------------------------- %
\newpage
\section{Residual Analysis for GLMs (Optional Section)}

\subsection{Pearson and Deviance Residuals} 
% https://v8doc.sas.com/sashtml/insight/chap39/sect55.htm





The \textbf{deviance residual} is the measure of deviance contributed from each observation and is given by
\[r_{Di} = \textrm{sign}( r_{i})
 \sqrt{ d_{i}}\]
where $d_i$ is the individual deviance contribution.
The deviance residuals can be used to check the model fit at each observation for generalized linear models. 

%These residuals are stored in variables named \textit{RD\_yname} for each response variable, where yname is the response variable name. 

The standardized and studentized deviance residuals are
\[
r_{Dsi} = \frac{r_{Di}}{\sqrt{\hat{ \phi} (1- h_{i})} }\]
\[r_{Dti} = \frac{r_{Di}}{\sqrt{ \hat{ \phi}_{(i)}
 (1- h_{i})}}\]
 
 


\end{document}
0