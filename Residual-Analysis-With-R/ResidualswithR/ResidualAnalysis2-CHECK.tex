

#### Why Use Residuals?}

If the model fit to the data were correct, the residuals would approximate the random errors that make the relationship between the explanatory variables and the response variable a statistical relationship. Therefore, if the residuals appear to behave randomly, it suggests that the model fits the data well. On the other hand, if non-random structure is evident in the residuals, it is a clear sign that the model fits the data poorly. The subsections listed below detail the types of plots to use to test different aspects of a model and give guidance on the correct interpretations of different results that could be observed for each type of plot.
%------------------------------------------------------------------------------------------------------------------------ %
\section{Introduction to Residuals}

The difference between the observed value of the dependent variable (y) and the predicted value ($\hat{y}$) is called the \textbf{residual} (e). Each data point has one residual.

\[\mbox{Residual} = \mbox{Observed value} - \mbox{Predicted value}\] 
\[e = y - \hat{y}\]

Both the sum and the mean of the residuals are equal to zero. 
%That is, Σ e = 0 and e = 0.


%-------------------------------------------------------------- %
<p>
\section{Residual Analysis for GLMs (Optional Section)}

#### Pearson and Deviance Residuals} 
% https://v8doc.sas.com/sashtml/insight/chap39/sect55.htm





The \textbf{deviance residual} is the measure of deviance contributed from each observation and is given by
\[r_{Di} = \textrm{sign}( r_{i})
 \sqrt{ d_{i}}\]
where $d_i$ is the individual deviance contribution.
The deviance residuals can be used to check the model fit at each observation for generalized linear models. 

%These residuals are stored in variables named \textit{RD\_yname} for each response variable, where yname is the response variable name. 

The standardized and studentized deviance residuals are
\[
r_{Dsi} = \frac{r_{Di}}{\sqrt{\hat{ \phi} (1- h_{i})} }\]
\[r_{Dti} = \frac{r_{Di}}{\sqrt{ \hat{ \phi}_{(i)}
 (1- h_{i})}}\]
 
 

