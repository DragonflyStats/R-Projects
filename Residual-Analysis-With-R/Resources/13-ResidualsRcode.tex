
\documentclass[residuals.tex]{subfiles}
\begin{document}
\Large

#### \texttt{resid} - Extracting Model Residuals}

% =========================================================%


* \texttt{residuals} is a generic function which extracts model residuals from objects returned by modeling functions. 

* The abbreviated form \texttt{resid} is an alias for residuals. It is intended to encourage users to access object components through an accessor function rather than by directly referencing an object slot. 

* All object classes which are returned by model fitting functions should provide a residuals method. (Note that the method is for \texttt{residuals} and not \texttt{resid}.) 

* Methods can make use of \texttt{naresid} methods to compensate for the omission of missing values. The default, nls and smooth.spline methods do. 

% =========================================================%


```{r}
residuals(fit)

resid(fit)



```{r}
residuals(fit1)

```

```{r}
> sum(residuals(fit))
[1] 1.096345e-15

> #Shapiro-Wilk Test for Normality
> shapiro.test(resid(fit))
 
 Shapiro-Wilk normality test
 
 data:  resid(fit)
 W = 0.9375, p-value = 0.06341
 
 



\subsubsection*{Weighted Residuals}

```{r}
x <- 1:10
w <- 0:9
y <- rnorm(x)
weighted.residuals(lmxy <- lm(y ~ x, weights = w))



\end{document}
