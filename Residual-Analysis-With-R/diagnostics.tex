
documentclass[a4paper,12pt]{article}
%%%%%%%%%%%%%%%%%%%%%%%%%%%%%%%%%%%%%%%%%%%%%%%%%%%%%%%%%
usepackage{eurosym}
usepackage{vmargin}
usepackage{amsmath}
usepackage{graphics}
usepackage{epsfig}
usepackage{subfigure}
usepackage{fancyhdr}
%usepackage{listings}
usepackage{framed}
usepackage{graphicx}

setcounter{MaxMatrixCols}{10}
%TCIDATA{OutputFilter=LATEX.DLL}
%TCIDATA{Version=5.00.0.2570}
%TCIDATA{<META NAME="SaveForMode" CONTENT="1">}
%TCIDATA{LastRevised=Wednesday, February 23, 2011 13:24:34}
%TCIDATA{<META NAME="GraphicsSave" CONTENT="32">}
%TCIDATA{Language=American English}

pagestyle{fancy}
setmarginsrb{20mm}{0mm}{20mm}{25mm}{12mm}{11mm}{0mm}{11mm}
lhead{MS4024} rhead{Mr. Kevin O'Brien}
chead{Numerical Computation}
%input{tcilatex}

begin{document}

<p>
### {Key Terms}

begin{ }
         * Residual: The difference between the predicted value (based on the regression equation) and the 
actual, observed value. 
         * Outlier: In linear regression, an outlier is an observation with large residual. In other words, it is an 
observation whose dependent-variable value is unusual given its value on the predictor variables. 
An outlier may indicate a sample peculiarity or may indicate a data entry error or other problem. 
         * Leverage: An observation with an extreme value on a predictor variable is a point with high 
leverage. Leverage is a measure of how far an independent variable deviates from its mean. High 
leverage points can have a great amount of effect on the estimate of regression coefficients. 
         * Influence: An observation is said to be influential if removing the observation substantially changes 
the estimate of the regression coefficients. Influence can be thought of as the product of leverage 
and outlierness. 
         * Cook's distance (or Cook's D): A measure that combines the information of leverage and residual of 
the observation. 
end{ }


%=================================================== %

end{document}
