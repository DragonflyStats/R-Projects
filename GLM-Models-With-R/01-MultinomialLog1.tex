\documentclass[00-GLMregslides.tex]{subfiles}

\begin{document}
	
\begin{frame}
\Large
	
\textbf{Multinomial Logistic Regression}\\
\bigskip
\begin{itemize}
\item Multinomial Logistic Regression is a classification method that generalizes logistic regression to \textbf{multiclass problems}, 
i.e. with more than two possible discrete outcomes.
\item That is, it is a model that is used to predict the probabilities of the different possible outcomes of a 
categorically distributed dependent variable, given a set of independent 
variables 
\item These predictor variables may be real-valued, binary-valued, categorical-valued, etc.
\end{itemize}
 
\end{frame}
	
\begin{frame}
\Large
	
\textbf{Multinomial Logistic Regression}\\
\bigskip
\begin{itemize}
\item Multinomial logistic regression is used to model nominal outcome variables, in which the log odds of the outcomes are modeled as a linear combination of the predictor variables.
\end{itemize}
 
\end{frame}




%================================================================%
\begin{frame}[fragile]
\frametitle{Multinomial Regression with \texttt{R}}
	\Large
	The main package we will use is the \textbf{\textit{nnet}} package. We will also use the \textbf{\textit{ggplot2}} and \textbf{\textit{reshape2}} package. 
	
	\begin{framed}
		\begin{verbatim}
		install.packages("nnet")
		library(nnet)
		
		library(ggplot2)
		library(reshape2)
		\end{verbatim}
	\end{framed}
	
\end{frame}
%=========================================%
\begin{frame}
	\textbf{nnet}
	
	\begin{itemize}
		\item Title : Feed-forward Neural Networks and Multinomial Log-Linear Models
		\item Description : Software for feed-forward neural networks with a single
		hidden layer, and for multinomial log-linear models.
		\item Authors : Brian Ripley ,William Venables 
		\item URL : http://www.stats.ox.ac.uk/pub/MASS4/
	\end{itemize}
	
\end{frame}
%========================================================%
\begin{frame}[fragile]		
	\frametitle{Multinomial Logistic Regression with \texttt{R} }	
	\Large	
		\textbf{Applications}
\begin{itemize}		
\item	Which major will a college student choose, given their grades, stated likes and dislikes, etc.?	
\item	Which blood type does a person have, given the results of various diagnostic tests?	
%\item	In a hands-free mobile phone dialing application, which person's name was spoken, given various properties of the speech signal?	
\item	Which candidate will a person vote for, given particular demographic characteristics?	
\item	Which country will a firm locate an office in, given the characteristics of the firm and of the various candidate countries?	
\end{itemize}		
\end{frame}		

%================================================================================================%
\begin{frame}[fragile]
	
	\frametitle{Multinomial Regression with \texttt{R}}
	\Large
	\textbf{Examples of Multinomial Logistic Regression}
	
	\begin{description}

		
		\item[Example 1.] Entering high school students make program choices among general program, vocational program and academic program. \\ Their choice might be modeled using their writing score and their social economic status.
	\end{description}
\end{frame}
%================================================================================================%
\begin{frame}[fragile]
	
	\frametitle{Multinomial Regression with \texttt{R}}
	\Large
	\textbf{Examples of Multinomial Logistic Regression}
	
	\begin{description}
		\item[Example 2.] People's occupational choices might be influenced by their parents' occupations and their own education level. \\ We can study the relationship of one's occupation choice with education level and father's occupation. \\ The occupational choices will be the outcome variable which consists of categories of occupations. 
		\end{description}
\end{frame}
%================================================================================================%
\begin{frame}[fragile]

\frametitle{Multinomial Regression with \texttt{R}}
\Large
\textbf{Examples of multinomial logistic regression}

\begin{description}

\item[Example 3.] A biologist may be interested in food choices that alligators make. Adult alligators might have different preferences from young ones. \\ The outcome variable here will be the types of food, and the predictor variables might be size of the alligators and other environmental variables.
 

\end{description}
\end{frame}

%================================================================================================%
%\begin{frame}[fragile]
%\Large
%\textbf{Description of the Data}
% 
%We will use the third example using the \textit{multilog} data set (or \textit{ml}).

% Let's first read in the data.

%\begin{framed}
%\begin{verbatim}
%
%ml <- read.csv("multilog.csv",header=T)
%\end{verbatim}
%\end{framed}
 
 
%\end{frame}
%================================================================================================%
\begin{frame}[fragile]
\Large	
	\textbf{Description of the Data}
\begin{itemize}
\item The outcome variable is \textbf{prog}, the program type. 
\item The two predictor variables are
\end{itemize}
{ \Large
\begin{enumerate}
 \item social economic status, \textbf{ses}, (a three-level categorical variable) 
 \item writing score, \textbf{write}, (a continuous variable).
\end{enumerate} 
}
\begin{itemize}
\item The data set contains variables on 200 students.
\end{itemize}
% Let's start with getting some descriptive statistics of the variables of interest.
\end{frame} 
%================================================================================================%
\begin{frame}[fragile]
\Large
\frametitle{Multinomial Regression with \texttt{R}}
\Large
\begin{framed}
\begin{verbatim}
table(ml$ses, ml$prog)

       general academic vocation
low         16       19       12
middle      20       44       31
high         9       42        7
\end{verbatim} 
\end{framed}
% with(ml, do.call(rbind, tapply(write, prog, function(x) c(M = mean(x), SD = sd(x)))))
\begin{verbatim} 
              M    SD
 general  51.33 9.398
 academic 56.26 7.943
 vocation 46.76 9.319
\end{verbatim} 

\end{frame}
%================================================================================================%
%\begin{frame}[fragile]
%
%\frametitle{Multinomial Regression with \texttt{R}}
%\Large
%Analysis methods you might consider
%\begin{itemize}
%\item Multinomial logistic regression, the focus of this page. 
%\item Multinomial probit regression, similar to multinomial logistic regression with independent normal error terms.
%\item Multiple-group discriminant function analysis. A multivariate method for multinomial outcome variables
%\item Multiple logistic regression analyses, one for each pair of outcomes: One problem with this approach is that each analysis is potentially run on a different sample. The other problem is that without constraining the logistic models, we can end up with the probability of choosing all possible outcome categories greater than 1.
%\end{itemize}
%\end{frame}
%================================================================================================%
%\begin{frame}[fragile]
%
%\frametitle{Multinomial Regression with \texttt{R}}
%\Large
%\begin{itemize}
%\item Collapsing number of categories to two and then doing a logistic regression: This approach suffers from loss of information and changes the original research questions to very different ones. 
%\item Ordinal logistic regression: If the outcome variable is truly ordered and if it also satisfies the assumption of proportional odds, then switching to ordinal logistic regression will make the model more parsimonious.
%\item Alternative-specific multinomial probit regression, which allows different error structures therefore allows to relax the IIA assumption. This requires that the data structure be choice-specific.
%\item Nested logit model, another way to relax the IIA assumption, also requires the data structure be choice-specific. 
%\end{itemize}
%\end{frame}
\end{document}
%================================================================================================%
