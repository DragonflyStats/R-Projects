
### Description of ToothGrowth dataset



str  (ToothGrowth {)}


    \begin{tcolorbox}[breakable, size=fbox, boxrule=1pt, pad at break*=1mm,colback=cellbackground, colframe=cellborder]
\prompt{In}{incolor}{ }{\hspace{4pt}}
\begin{Verbatim}[commandchars=\\\{\}]
\PY{k+kp}{unique}  (ToothGrowth$dose {)}
\PY{c+c1}{\PYZsh{}\PYZsh{} [1] 0.5 1.0 2.0}
\end{Verbatim}
\end{tcolorbox}



\subsubsection{Exploring distributions}\label{exploring-distributions}

    \begin{tcolorbox}[breakable, size=fbox, boxrule=1pt, pad at break*=1mm,colback=cellbackground, colframe=cellborder]
\prompt{In}{incolor}{3}{\hspace{4pt}}
\begin{Verbatim}[commandchars=\\\{\}]
\PY{k+kn}{library}  (ggplot2 {)}
\PY{c+c1}{\PYZsh{}bar plot of counts of dose by supp}
\PY{c+c1}{\PYZsh{}data are balanced, so not so interesting}
ggplot  (ToothGrowth {,} aes  (x  = dose {,} fill  = supp {)} {)}  +}  geom\PYZus{}bar  ( {)}
\end{Verbatim}
\end{tcolorbox}


    Statistical models often make assumptions about the distribution of the
outcome (or its residuals), so an inspection might be wise. First let's
check the overall distribution of ``len'' with a density plot:

    \begin{tcolorbox}[breakable, size=fbox, boxrule=1pt, pad at break*=1mm,colback=cellbackground, colframe=cellborder]
\prompt{In}{incolor}{7}{\hspace{4pt}}
\begin{Verbatim}[commandchars=\\\{\}]
\PY{c+c1}{\PYZsh{}density of len}
ggplot  (ToothGrowth {,} aes  (x  = len {)} {)}   +} geom\PYZus{}density  ( {)}
\end{Verbatim}
\end{tcolorbox}

    
    
    \begin{center}
    \adjustimage{max size={0.9\linewidth}{0.9\paperheight}}{output_7_1.png}
    \end{center}
    { \hspace*{\fill} \\}
    
    We can get densities of len by supp by mapping supp to color:

    \begin{tcolorbox}[breakable, size=fbox, boxrule=1pt, pad at break*=1mm,colback=cellbackground, colframe=cellborder]
\prompt{In}{incolor}{4}{\hspace{4pt}}
\begin{Verbatim}[commandchars=\\\{\}]
\PY{k+kn}{library}  (ggplot2 {)}
\PY{c+c1}{\PYZsh{}density of len by supp}
ggplot  (ToothGrowth {,} aes  (x  = len {,} color  = supp {)} {)}  +} 
  geom\PYZus{}density  ( {)}
\end{Verbatim}
\end{tcolorbox}

    
    
    \begin{center}
    \adjustimage{max size={0.9\linewidth}{0.9\paperheight}}{output_9_1.png}
    \end{center}
    { \hspace*{\fill} \\}
    
    \paragraph{Summarizing the outcome across a
predictor}\label{summarizing-the-outcome-across-a-predictor}

Because dose takes on only 3 values, many point are crowded in 3
columns, obscuring the shape of relationship between dose and len. We
replace the column of points at each dose value with its mean and
confidence limit using stat\_summary instead of geom\_point.

    The outcome distributions appear a bit skewed, but the samples are
small.

\subsubsection{Quick scatterplot of outcome and
predictor}\label{quick-scatterplot-of-outcome-and-predictor}

We plot the dose-tooth length (len) relationship.

    \begin{tcolorbox}[breakable, size=fbox, boxrule=1pt, pad at break*=1mm,colback=cellbackground, colframe=cellborder]
\prompt{In}{incolor}{5}{\hspace{4pt}}
\begin{Verbatim}[commandchars=\\\{\}]
\PY{c+c1}{\PYZsh{}not the best scatterplot}
tp  \PYZlt{}\PYZhy{}} ggplot  (ToothGrowth {,} aes  (x  = dose {,} y  = len {)} {)}
tp  +} geom\PYZus{}point  ( {)}
\end{Verbatim}
\end{tcolorbox}

    
    
    \begin{center}
    \adjustimage{max size={0.9\linewidth}{0.9\paperheight}}{output_12_1.png}
    \end{center}
    { \hspace*{\fill} \\}
    
    \begin{tcolorbox}[breakable, size=fbox, boxrule=1pt, pad at break*=1mm,colback=cellbackground, colframe=cellborder]
\prompt{In}{incolor}{6}{\hspace{4pt}}
\begin{Verbatim}[commandchars=\\\{\}]
\PY{c+c1}{\PYZsh{}mean and cl of len at each dose}
tp.1  \PYZlt{}\PYZhy{}} tp  +} stat\PYZus{}summary  (fun.data  = \PY{l+s}{\PYZdq{}}\PY{l+s}{mean\PYZus{}cl\PYZus{}normal\PYZdq{}} {)}
tp.1
\end{Verbatim}
\end{tcolorbox}

    
    
    \begin{center}
    \adjustimage{max size={0.9\linewidth}{0.9\paperheight}}{output_13_1.png}
    \end{center}
    { \hspace*{\fill} \\}
    
    An additional call to stat\_summary with fun.y=mean (fun.y because mean
returns one value) and changing the geom to ``line'' adds a line between
means. ​

    \begin{tcolorbox}[breakable, size=fbox, boxrule=1pt, pad at break*=1mm,colback=cellbackground, colframe=cellborder]
\prompt{In}{incolor}{7}{\hspace{4pt}}
\begin{Verbatim}[commandchars=\\\{\}]
\PY{c+c1}{\PYZsh{}add a line plot of means to see dose\PYZhy{}len relationship}
\PY{c+c1}{\PYZsh{}maybe not linear}
tp.2  \PYZlt{}\PYZhy{}} tp.1  +} stat\PYZus{}summary  (fun.y  = \PY{l+s}{\PYZdq{}}\PY{l+s}{mean\PYZdq{}} {,} geom  = \PY{l+s}{\PYZdq{}}\PY{l+s}{line\PYZdq{}} {)}
tp.2
\end{Verbatim}
\end{tcolorbox}

    
    
    \begin{center}
    \adjustimage{max size={0.9\linewidth}{0.9\paperheight}}{output_15_1.png}
    \end{center}
    { \hspace*{\fill} \\}
    
    \begin{tcolorbox}[breakable, size=fbox, boxrule=1pt, pad at break*=1mm,colback=cellbackground, colframe=cellborder]
\prompt{In}{incolor}{ }{\hspace{4pt}}
\begin{Verbatim}[commandchars=\\\{\}]
Does a third variable moderate the x \PYZhy{}}y relationship ?}

Does the dose \PYZhy{}}len relationship depend on supp ?} We can specify new global aesthetics \PY{k+kr}{in} aes.
\end{Verbatim}
\end{tcolorbox}

    \begin{tcolorbox}[breakable, size=fbox, boxrule=1pt, pad at break*=1mm,colback=cellbackground, colframe=cellborder]
\prompt{In}{incolor}{ }{\hspace{4pt}}
\begin{Verbatim}[commandchars=\\\{\}]
\PY{c+c1}{\PYZsh{}all plots in tp.2 will now be colored by supp}
tp.2  +} aes  (color  = supp {)}
\end{Verbatim}
\end{tcolorbox}

    \subsubsection{Interpreting the previous
graph}\label{interpreting-the-previous-graph}

This graph suggests:

\begin{enumerate}
\def\labelenumi{\arabic{enumi}.}

* 
  The slope of the dose-response curve decreases as dose increases for
  both supp types, suggesting a quadratic function.
* 
  The slope of the OJ curve flattens more dramatically, perhaps
  suggesting the quadratic term is different between supplement groups
* 
  The initial slopes look rather similar, perhaps suggesting that the
  linear term may not be different between groups
* 
  The 2 supp group means differ at the two lower doses, but not at the
  highest dose
\end{enumerate}

ggplot2 makes graphs summarizing the outcome easy ​ We just plotted
means and confidence limits of len, with lines connecting the means,
separated by supp, all without any manipulation to the original data! ​
The stat\_summary function facilitates looking at patterns of means,
just as regression models do. ​ Next we fit our linear regression model
and check model assumptions with diagnostic graphs. ​

    \subsubsection{Model preliminaries}\label{model-preliminaries}

​ We want to model how tooth length (len) is predicted by dose, allowing
for moderation of this relationship through an interaction with supp. ​
We assume that dose and tooth length have a smooth, continuous
relationship in the range of doses tested, so we will treat dose as a
continuous (numeric) predictor. We also create a dose-squared variable,
for use in the quadratic model and prediction later. ​

    \begin{tcolorbox}[breakable, size=fbox, boxrule=1pt, pad at break*=1mm,colback=cellbackground, colframe=cellborder]
\prompt{In}{incolor}{8}{\hspace{4pt}}
\begin{Verbatim}[commandchars=\\\{\}]
\PY{c+c1}{\PYZsh{}create dose\PYZhy{}squared variable}
ToothGrowth$dosesq  \PYZlt{}\PYZhy{}} ToothGrowth$dose \PYZca{}}\PY{l+m}{2}
\end{Verbatim}
\end{tcolorbox}

    \subsubsection{Fitting the model}\label{fitting-the-model}

We noticed in the previous graph that the dose-len relationship appears
quadratic, that the quadratic effect may differ between supp groups, but
that the initial linear slope may not be so different. We fit a model to
reflect our expectations:

    \begin{tcolorbox}[breakable, size=fbox, boxrule=1pt, pad at break*=1mm,colback=cellbackground, colframe=cellborder]
\prompt{In}{incolor}{9}{\hspace{4pt}}
\begin{Verbatim}[commandchars=\\\{\}]
lm2  \PYZlt{}\PYZhy{}} lm  (len  \PYZti{}} dose  +} dosesq *}supp {,} data  = ToothGrowth {)}
\PY{k+kp}{summary}  (lm2 {)}$coef
\PY{c+c1}{\PYZsh{}\PYZsh{}                 Estimate Std. Error    t value     Pr(\PYZgt{}|t|)}
\PY{c+c1}{\PYZsh{}\PYZsh{} (Intercept)    0.7491667  2.7983895  0.2677135 7.899213e\PYZhy{}01}
\PY{c+c1}{\PYZsh{}\PYZsh{} dose          30.1550000  5.2474684  5.7465806 4.114588e\PYZhy{}07}
\PY{c+c1}{\PYZsh{}\PYZsh{} dosesq        \PYZhy{}8.7238095  2.0402571 \PYZhy{}4.2758383 7.640686e\PYZhy{}05}
\PY{c+c1}{\PYZsh{}\PYZsh{} suppVC        \PYZhy{}6.4783333  1.3762287 \PYZhy{}4.7073088 1.739152e\PYZhy{}05}
\PY{c+c1}{\PYZsh{}\PYZsh{} dosesq:suppVC  1.5876190  0.5770719  2.7511635 8.024694e\PYZhy{}03}
\end{Verbatim}
\end{tcolorbox}

    \begin{tabular}{r|llll}
  & Estimate & Std. Error & t value & Pr(>\textbar{}t\textbar{})\\
\hline
	(Intercept) &  0.7491667   & 2.7983895    &  0.2677135   & 7.899213e-01\\
	dose & 30.1550000   & 5.2474684    &  5.7465806   & 4.114588e-07\\
	dosesq & -8.7238095   & 2.0402571    & -4.2758383   & 7.640686e-05\\
	suppVC & -6.4783333   & 1.3762287    & -4.7073088   & 1.739152e-05\\
	dosesq:suppVC &  1.5876190   & 0.5770719    &  2.7511635   & 8.024694e-03\\
\end{tabular}


    
    The model appears to conform to our expectations.

    \subsubsection{Example 1: Model
diagnostics}\label{example-1-model-diagnostics}

Statistical inference depends on the assumptions of the regression
model, which we check with diagnostic graphs. Common diagnostics for
linear regression



* 
  inspect the normality of the residuals
* 
  verify that the residuals show no trends (assumption of linearity) and
  are homoscedastic,
* 
  check for overly influential outliers.
* 
  fortify makes linear regression diagnostics easy


    Conveniently, the fortify function takes a lm model object (one among
several classes) and returns a dataset with several estimated diagnostic
variables including:



* 
  .hat: leverages(influence)
* 
  .sigma: residual standard deviation when observation dropped from
  model
* 
  .cooksd: Cook's distance
* 
  .fitted: fitted (predicted) values
* 
  .resid: residuals
* 
  .stdresid: standardized residuals


We fortify our lm2 object with these diagnostic variables and take a
quick look a the new variables.

    \begin{tcolorbox}[breakable, size=fbox, boxrule=1pt, pad at break*=1mm,colback=cellbackground, colframe=cellborder]
\prompt{In}{incolor}{11}{\hspace{4pt}}
\begin{Verbatim}[commandchars=\\\{\}]
\PY{c+c1}{\PYZsh{}create dataset with original data and diagnostic variables}
lm2diag  \PYZlt{}\PYZhy{}} fortify  (lm2 {)}
\PY{c+c1}{\PYZsh{}quick look}
\PY{k+kp}{head}  (lm2diag {)}
\end{Verbatim}
\end{tcolorbox}

    \begin{tabular}{r|llllllllll}
 len & dose & dosesq & supp & .hat & .sigma & .cooksd & .fitted & .resid & .stdresid\\
\hline
	  4.2         & 0.5          & 0.25         & VC           & 0.08095238   & 3.623134     & 0.0165458459 & 7.564286     & -3.3642857   & -0.96913367 \\
	 11.5         & 0.5          & 0.25         & VC           & 0.08095238   & 3.611516     & 0.0226438547 & 7.564286     &  3.9357143   &  1.13374237 \\
	  7.3         & 0.5          & 0.25         & VC           & 0.08095238   & 3.654279     & 0.0001021058 & 7.564286     & -0.2642857   & -0.07613152 \\
	  5.8         & 0.5          & 0.25         & VC           & 0.08095238   & 3.645881     & 0.0045503109 & 7.564286     & -1.7642857   & -0.50822934 \\
	  6.4         & 0.5          & 0.25         & VC           & 0.08095238   & 3.650733     & 0.0019816291 & 7.564286     & -1.1642857   & -0.33539021 \\
	 10.0         & 0.5          & 0.25         & VC           & 0.08095238   & 3.638080     & 0.0086727319 & 7.564286     &  2.4357143   &  0.70164455 \\
\end{tabular}


    
    \subsubsection{Normality of residuals: q-q plot and
stat\_qq}\label{normality-of-residuals-q-q-plot-and-statux5fqq}


* 
  A q-q plot can assess the assumption that the residuals are normally
  distributed by plotting the standardized residuals (observed z-score)
  against theoretical quantiles of the normal distribution (expected
  z-score if normally distributed).
* 
  stat\_qq creates a qq-plot. The only required aesthetic is sample,
  which we map to the standardized residual variable created by fortify,
  .stdresid.
* 
  A diagonal reference line (intercept=0, slope=1) is added to the plot
  with geom\_abline, representing perfect fit to a normal distribution.
  The normal distribution will appear to be a reasonable fit below.


    \begin{tcolorbox}[breakable, size=fbox, boxrule=1pt, pad at break*=1mm,colback=cellbackground, colframe=cellborder]
\prompt{In}{incolor}{12}{\hspace{4pt}}
\begin{Verbatim}[commandchars=\\\{\}]
\PY{c+c1}{\PYZsh{}q\PYZhy{}q plot of residuals and diagonal reference line}
\PY{c+c1}{\PYZsh{}geom\PYZus{}abline default aesthetics are yintercept=0 and slope=1}
ggplot  (lm2diag {,} aes  (sample  = \PY{l+m}{.}stdresid {)} {)}  +} 
  stat\PYZus{}qq  ( {)}  +} 
  geom\PYZus{}abline  ( {)}
\end{Verbatim}
\end{tcolorbox}

    
    
    \begin{center}
    \adjustimage{max size={0.9\linewidth}{0.9\paperheight}}{output_28_1.png}
    \end{center}
    { \hspace*{\fill} \\}
    
    \begin{tcolorbox}[breakable, size=fbox, boxrule=1pt, pad at break*=1mm,colback=cellbackground, colframe=cellborder]
\prompt{In}{incolor}{ }{\hspace{4pt}}
\begin{Verbatim}[commandchars=\\\{\}]
\PY{c+c1}{\PYZsh{}\PYZsh{}\PYZsh{} Linearity and Homoscedasticity: residuals vs fitted}

We \PY{k+kr}{next} assess the assumptions of homoscedescasticity and linear relationships between the outcome and predictors. A residuals vs fitted   (predicted value {)} plot assesses both of these assmuptions.

An even spread of residuals around \PY{l+m}{0} suggests homoscedasticity {,} and a zero {,} flat slope \PY{k+kr}{for} residuals against fitted suggests linearity of predictor effects.
\end{Verbatim}
\end{tcolorbox}

    \begin{tcolorbox}[breakable, size=fbox, boxrule=1pt, pad at break*=1mm,colback=cellbackground, colframe=cellborder]
\prompt{In}{incolor}{ }{\hspace{4pt}}
\begin{Verbatim}[commandchars=\\\{\}]
We build our residuals vs fitted plot like so :}

\PY{l+m}{1}\PY{l+m}{.} Start with a scatter plot of x  = \PY{l+m}{.}fitted and y  = \PY{l+m}{.}stdresid.
\PY{l+m}{2}\PY{l+m}{.} Add a plot the means and standard errors of the residuals across fitted values using stat\PYZus{}summary. The standard error bars somewhat address homoskedasticity.
\PY{l+m}{3}\PY{l+m}{.} Plot a line representing the mean trend of the residuals with another call to stat\PYZus{}summary   (changing \PY{k+kr}{function} to mean and geom to line {)}\PY{l+m}{.} Normally {,} we would use geom\PYZus{}smooth to plot a loess curve to visualize the trend among residuals {,} but loess smooths do not work well when the variable mapped to x is discrete.
\PY{l+m}{4}\PY{l+m}{.} The error bars do not appear to vary too much and the line seems sufficiently flat to feel that neither assumption has been violated.
\end{Verbatim}
\end{tcolorbox}

    \begin{tcolorbox}[breakable, size=fbox, boxrule=1pt, pad at break*=1mm,colback=cellbackground, colframe=cellborder]
\prompt{In}{incolor}{ }{\hspace{4pt}}
\begin{Verbatim}[commandchars=\\\{\}]
\PY{c+c1}{\PYZsh{}residual vs fitted, means and s.e.}
ggplot  (lm2diag {,} aes  (x  = \PY{l+m}{.}fitted {,} y  = \PY{l+m}{.}stdresid {)} {)}  +} 
  geom\PYZus{}point  ( {)}  +} 
  stat\PYZus{}summary  ( {)}  +} 
  stat\PYZus{}summary  (fun.y  = \PY{l+s}{\PYZdq{}}\PY{l+s}{mean\PYZdq{}} {,} geom  = \PY{l+s}{\PYZdq{}}\PY{l+s}{line\PYZdq{}} {)}
\PY{c+c1}{\PYZsh{}\PYZsh{} No summary function supplied, defaulting to `mean\PYZus{}se()}
\end{Verbatim}
\end{tcolorbox}

    \subsubsection{Identifying influential
observations}\label{identifying-influential-observations}

Strongly influential observations can distort regression coefficients.
The most influential datapoints will typically have more extreme
predictor values (leverage), measured by .hat, and large residuals. The
overall influence of an observation on the model is measured by Cook's
D, variable .cooksd.

The Toothgrowth dataset is fairly balanced across doses and supps (20 at
each of 3 doses, 30 at each of 2 supp). Thus, no values are ``extreme'',
so influential observations for this model will be those with large
residuals.

In the following plot: * map .hat to x, .stdresid to y * map .cooksd to
size, making more influential points larger. * label the points in
geom\_text with their row numbers for identification

The dependence of Cook's D on both leverage and residual is apparent in
the plot, with Cook's D rising as we move away from center (larger
residual) and to the right (larger leverage). No point looks too
influential for concern.

    \begin{tcolorbox}[breakable, size=fbox, boxrule=1pt, pad at break*=1mm,colback=cellbackground, colframe=cellborder]
\prompt{In}{incolor}{13}{\hspace{4pt}}
\begin{Verbatim}[commandchars=\\\{\}]
\PY{c+c1}{\PYZsh{} in geom\PYZus{}text we SET size to 4 so that size of text is constant,}
\PY{c+c1}{\PYZsh{}   and not tied to .cooksd.  We also  nudge the text .001 (x\PYZhy{}axis units)}
\PY{c+c1}{\PYZsh{}   to the left, and separate overlapping labels}
ggplot  (lm2diag {,} aes  (x  = \PY{l+m}{.}hat {,} y  = \PY{l+m}{.}stdresid {,} size  = \PY{l+m}{.}cooksd {)} {)}  +}     
  geom\PYZus{}point  ( {)}  +}
  geom\PYZus{}text  (aes  (label  = \PY{k+kp}{row.names}  (lm2diag {)} {)} {,} 
             size  = \PY{l+m}{4} {,} nudge\PYZus{}x  = \PY{l+m}{\PYZhy{}0.003} {,} check\PYZus{}overlap  = \PY{n+nb+bp}{T} {)}
\end{Verbatim}
\end{tcolorbox}

    
    
    \begin{center}
    \adjustimage{max size={0.9\linewidth}{0.9\paperheight}}{output_33_1.png}
    \end{center}
    { \hspace*{\fill} \\}
    
    \begin{tcolorbox}[breakable, size=fbox, boxrule=1pt, pad at break*=1mm,colback=cellbackground, colframe=cellborder]
\prompt{In}{incolor}{ }{\hspace{4pt}}
\begin{Verbatim}[commandchars=\\\{\}]

\end{Verbatim}
\end{tcolorbox}


    % Add a bibliography block to the postdoc
    
    
    
    \end{document}
